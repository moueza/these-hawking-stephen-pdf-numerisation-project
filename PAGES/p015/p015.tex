\documentclass{article}
\usepackage{hyperref}
\usepackage[USenglish]{babel}
%\usepackage{csquotes}
%\usepackage{dirtytalk}
%https://perso.imt-mines-albi.fr/~gaborit/latex/latex-in-french.html
%\usepackage[english,francais]{babel}
%\usepackage[english,french]{babel}


%https://tex.stackexchange.com/questions/248788/the-very-basics-of-french-accents
%\documentclass[12pt]{amsart}

\usepackage[utf8]{inputenc}

%entiers sets https://texblog.org/2007/08/27/number-sets-prime-natural-integer-rational-real-and-complex-in-latex/
%\usepackage{amsfonts} 
% or 
\usepackage{amssymb}

%Majuscules accentuees https://www.xm1math.net/doculatex/caracteres_speciaux.html
\usepackage{eurosym}
\usepackage{systeme}
\usepackage{enumerate}
%numerotation https://www.xm1math.net/doculatex/structure.html
\renewcommand{\thesubsection}{\Roman{subsection}}

%\title{\textbf{PROPERTIES OF EXPANDING UNIVERSES.}}


%https://stackoverflow.com/questions/4262294/remove-default-date-in-latex-article

\begin{document}


%https://perso.imt-mines-albi.fr/~gaborit/latex/latex-in-french.html
%\renewcommand{\contentsname}{Sommaire}


%https://tex.stackexchange.com/questions/59460/custom-title-page-in-report-or-book-class
%\textcolor{other}{TRINITY HALL}





CHAPTER 1

The Hoyle-Narlikar Theory

of Gravitation

1. Introduction

The success of Maxwell's equations has led to

15 of 134 MS-PHD-O... 5 MS-PHD-O... 6 MS-PHD-O... 7 MS-PHD-O... 8 MS-PHD-O... 9 MS-PHD-O...

MS-PHD-O... 11 MS-PHD-O... 12 MS-PHD-O... 13 MS-PHD-O... 14 MS-PHD-O...

15 MS-PHD-O... 16 MS-PHD-O... 17 MS-PHD-O... 18 MS-PHD-O... 19 MS-PHD-O... 20

MS-PHD-O... 21 MS-PHD-O... 22 MS-PHD-O... 23 MS-PHD-O... 24 MS-PHD-O... 25

MS-PHD-O... MS-PHD-O... 27 MS-PHD-O... 28 MS-PHD-O... 29 MS-PHD-O... 30 MS-PHD-O...

31 MS-PHD-O... 32 MS-PHD-O... 33 MS-PHD-O... 34 MS-PHD-O... 35 MS-PHD-O... 36

MS-PHD-O... 37 MS-PHD-O... 38 MC-DHD-O 30

electrodynamics being normally formulated in terms of

fields that have degrees of freedom independent of the

particles in them. However, Gauss suggested that an

action-at-a-distance theory in which the action travelled

at a finite velocity might be possible. This idea was

developed by Wheeler and

Feynman (1,2) who derived their theory from an action-principle

that involved only direct interactions between pairs of

particles. A feature of this theory was that the

'pseudo'-fields introduced are the half-retarded plus

half-advanced fields claculated from the world-lines of the

particles. However, Wheeler and Feynman, and, in a

different way, Hogarth were able to show that, provided

certain cosmological conditions were satisfied, these

fields could combine to give the observed field. Hoyle and

Narlikar (4) extended the theory to general space-times and

obtained similar theories for their 'C'-fiela 5and for the

gravitational field (6). It is with these theories that

this chapter is concerned.





\footnote{Written by Peter MOUEZA}
\end{document}
