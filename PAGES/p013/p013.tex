\documentclass{article}
\usepackage{hyperref}
\usepackage[USenglish]{babel}
%\usepackage{csquotes}
%\usepackage{dirtytalk}
%https://perso.imt-mines-albi.fr/~gaborit/latex/latex-in-french.html
%\usepackage[english,francais]{babel}
%\usepackage[english,french]{babel}


%https://tex.stackexchange.com/questions/248788/the-very-basics-of-french-accents
%\documentclass[12pt]{amsart}

\usepackage[utf8]{inputenc}

%entiers sets https://texblog.org/2007/08/27/number-sets-prime-natural-integer-rational-real-and-complex-in-latex/
%\usepackage{amsfonts} 
% or 
\usepackage{amssymb}

%Majuscules accentuees https://www.xm1math.net/doculatex/caracteres_speciaux.html
\usepackage{eurosym}
\usepackage{systeme}
\usepackage{enumerate}
%numerotation https://www.xm1math.net/doculatex/structure.html
\renewcommand{\thesubsection}{\Roman{subsection}}

%\title{\textbf{PROPERTIES OF EXPANDING UNIVERSES.}}


%https://stackoverflow.com/questions/4262294/remove-default-date-in-latex-article
\date{1st february 1966}

\begin{document}


%https://perso.imt-mines-albi.fr/~gaborit/latex/latex-in-french.html
%\renewcommand{\contentsname}{Sommaire}


%https://tex.stackexchange.com/questions/59460/custom-title-page-in-report-or-book-class
%\textcolor{other}{TRINITY HALL}






and contain matter whose pressure is creater than minus onethird

the density. An ovent horizon exists when there are events that

a given observer will never see. The steadystate model (K0 Rce )

is an example of one with an event horizon. forizons will be

further discussed in Chapter 4 which also deals with the

occurrence of singularities of spacetime and their connection

with topology.

\paragraph{}
Each chapter is self-contained and has its own references. The

following notation is used throughout; space-time is taken to be

a Riemannian manifold with metric tensor ge; This is taken to

have signature -2 except in Chapter 2 where, in order to

facilitate comparison with previous work, the signature is 2.

Covariant aifferentiation is indicated by a semi-colon. Units

are employed in wicho, the speed of 11cht,and K, the

seavitational constant, equal one.


\footnote{Written by Peter MOUEZA}
\end{document}
