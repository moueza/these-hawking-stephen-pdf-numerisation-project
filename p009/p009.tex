\documentclass{article}
\usepackage{hyperref}
\usepackage[USenglish]{babel}
%https://perso.imt-mines-albi.fr/~gaborit/latex/latex-in-french.html
%\usepackage[english,francais]{babel}
%\usepackage[english,french]{babel}


%https://tex.stackexchange.com/questions/248788/the-very-basics-of-french-accents
%\documentclass[12pt]{amsart}

\usepackage[utf8]{inputenc}

%entiers sets https://texblog.org/2007/08/27/number-sets-prime-natural-integer-rational-real-and-complex-in-latex/
%\usepackage{amsfonts} 
% or 
\usepackage{amssymb}

%Majuscules accentuees https://www.xm1math.net/doculatex/caracteres_speciaux.html
\usepackage{eurosym}
\usepackage{systeme}
\usepackage{enumerate}
%numerotation https://www.xm1math.net/doculatex/structure.html
\renewcommand{\thesubsection}{\Roman{subsection}}

\title{\textbf{PROPERTIES OF EXPANDING UNIVERSES.}}


%https://stackoverflow.com/questions/4262294/remove-default-date-in-latex-article
\date{1st february 1966}

\begin{document}


%https://perso.imt-mines-albi.fr/~gaborit/latex/latex-in-french.html
%\renewcommand{\contentsname}{Sommaire}


%https://tex.stackexchange.com/questions/59460/custom-title-page-in-report-or-book-class
%\textcolor{other}{TRINITY HALL}

\part{\textbf{ABSTRACT}}


Some implications and consequences of the expansion of the universe are examined. In \textbf{Chapter 1} it is shown that this expansion creates grave difficulties for the Hoyle-Narlikar theory of gravitation. \textbf{Chapter 2} deals with perturbations of an expanding homogeneous and isotropic universe. The \textbf{conclusion} is reached that galaxies cannot be formed as a result of the growth of perturbations that were initially small.




\footnote{Written by Peter MOUEZA}
\end{document}
